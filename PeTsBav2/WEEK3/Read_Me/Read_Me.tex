\documentclass[a4paper,10pt]{article}
\usepackage[utf8]{inputenc}

%opening
\title{WEEK3: Skyrme Potential}
\author{Ni Fang\\ Kai Wang\\Claudia Gonzalez-Boquera}
\date{TALENT 2016}
\begin{document}

\maketitle
\section{Structure}
This program calculates the eigenvalues and wavefunctions of each given nlj orbital when having a t0-t3 Skyrme potential.

The program consists in a self-consistent method to calculate the levels of a given nucleus.

We start from wavefunctions and densities trials (using the WS 3D code).  
When having this, we calculate the self-consistent loop.
We start calculating the full series of levels for the nucleus we want to study.
As in the previous codes, the eigenvalues are found with the bissection method, which includes the numerov 
method in order to calculate the eigenfunction for each eigenvalue. For each
eigenfunction found, we calculate its nodes, and we modify the extremes of the 
bissection method according to them. When the extremes are closer than epsi,
we exit the loop, and put the eigenvalue as the right extreme of the bissection
method. 
Then we recalculate the eigenfunction with the shooting method, with 
right hand side equal to 0, in order to find the convergence, and we normalize it. 

We do this for neutrons and for protons separately. 

When all possible orbitals are calculated, we rearrange the energy in ascending order, 
in order to find the correct order in the levels. 

After, the density profiles are calculated.

When having all of this, we calculate and compare the Hartree-Fock total energy with the one obtained in the 
previous loop. 
If the difference is less than epsilon, we exit the loop, as we have reached the convergence.

\subsection{Functions and Subroutines}
\begin{itemize}

\item Subroutine WoodsSaxon3D(energy$\_$n,energy$\_$p,quant$\_$n,quant$\_$p,wf$\_$n, wf$\_$p): subroutine to get the 
trial energies and wavefunctions. The energies of the neutrons (protons) will be
stored in the array energy$\_$n  (energy$\_$p), the neutron (proton) quantum numbers
in quant$\_$n (quant$\_$p) and the wavefunctions in wf$\_$n (wf$\_$p). In order to arrange the energy 
levels, we use the subroutine arrage$\_$energy(matrix,wavef).
\item Subroutine arrange$\_$energy(matrix,wavef): function that arranges the eigenvalues in ascending order. 
These values are inside matrix, which includes the energies and quantum numbers. All is 
ordered following the energy ascendancy. Also, the ,matrix that includes all the wavefunctions
(wavef) is also ordered.
 \item Subroutine simpson(N, dx, f, res) : subroutine to integrate the densities, in order to normalize the 
eigenfunctions. Inputs: N number of points in the mesh, dx space between the points in the mesh,f array
to integrate. Output:res  value of the integration. 
\item Function Vcoulomb (N, rho$\_$p): function to calculate the Coulomb potential for protons. As an input
the position in the mesh to calculate (N) and the proton density (rho$\_$p).
\end{itemize}

\section{Files}
\begin{itemize}
 \item global.f90: module that includes the global variables.
 \item initial$\_$phi.f90: file that contains the WoodsSaxon3D subroutine to find the initial eigenvalues and wavefunctions.
 \item nucleon.txt: input file (see below).
 \item Coulomb.f90: file that includes the function Vcoulomb to calculate the value of the Coulomb potential. 
 \item main.f90: main file that includes the code to find the eigenvalues and eigenfunctions in a self-consistent way. 
 \item Makefile : makefile to compile the code (see below). 
 \item simpson.f90: file that contains the subroutine that integrates using the Simpson method.
 \item arrange.f90: file that contains the subroutine that arranges the energy in ascending order for the Hartree-Fock eigenvalues. 
  \item arrange2.f90: file that contains the subroutine that arranges the energy in ascending order for the WoodsSaxon3D subroutine. 
 \item data$\_$plot.plt: file to plot with gnuplot the proton and neutron densities. 
 \item data$\_$plot2.plt: file to plot with gnuplot the proton and neutron kinetic densities. 
  \end{itemize}


\section{Input file---nucleon.txt}
\begin{itemize}
 \item proton, neutron = number of protons and neutron of the nucleus to study. 
  \item h2m = value of $\hbar^2/2m$.
   \item R$_\mathrm{max}$ = size of our considered and studied system.
    \item N= number of points in the mesh.
    \item charge= not used
 \item epsi= epsilon error to find the solution. 
 \item E$_\mathrm{m}$= left value of the energy when finding the eigenvalues using the bissection method.
 \item E$_\mathrm{p}$ = right value of the energy when finding the eigenvalues using the bissection method.
 \item e2= value of e$^2$, value of the charge for the Coulomb potential.
 \item out$\_$wave$\_$func = logical parameter to output all the wavefunctions calculated in files or not.
 \item n$\_$max, l$\_$max = maximum n and l quantum numbers to calculate.
 \item r0 = parameter to calculate R0 in Wood-Saxon potential. 
 \item a = difusivity of the Wood-Saxon potential. 
 \item CMcorrection = logical parameter to indicate if CM correction is used.

\end{itemize}

\section{Output}
\begin{itemize}
 \item Energy levels of neutrons (energy$\_$level$\_$n.dat) and protons (energy$\_$level$\_$p.dat).
 \item Density profiles of neutrons (density$\_$n$\_$new.dat) and protons(density$\_$p$\_$new.dat).
 \item If wanted, output with all wavefunctions calculated. (output$\_$wave$\_$func =.true.)
 \item Console: iteration, nucleon number, total energy of the system, single particle energy, kinetic energy, 
 Skyrme energy, Coulomb energy.
 \item plots of the neutron and proton densities and kinetic densities.
\end{itemize}


\section{Makefile}
To run the program, there is a Makefile, that compiles all files that form the code. 
To compile, type \textbf{make}, and an executable \textbf{main} is created. To run the code, 
type \textbf{./main}. And to delete all exectuables and outputs, type either \textbf{make clean} or
\textbf{make clear}. Makefile has to be modified with vi or vim programs.

\section{Extras}
\subsection{Coulomb potential}
Coulomb potential has been added, but the total energy does not give the correct value. \textbf{It is not correct. It has to 
be modified}.
\subsection{BCS.}
With the t0-t3 Skyrme code, we have added, in a different folder, the BCS calculation. 
New subroutines: solve$\_$BCS$\_$equation(L,E$\_$data,N,E$\_$tot, lambda, Delta, v): inputs:  L: the number of shell
E$\_$data(1:L,1:4): 1:energy, 2:n, 3:angular momentum, 4:spin,
N: particle number in BCS system output,
 E$\_$tot: total energy in BCS system,
 lambda: chemical potential, 
 Delta: pairing gap, v(1:L): occupation probability in BCS system..
 Subroutine LU: subroutine that implements the LU decomposition to solve $df/dE * dE = -f$.
 
 \subsection{Full Skyrme}
 Not finished. 
\end{document}
