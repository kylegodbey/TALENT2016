\documentclass[a4paper,10pt]{article}
\usepackage[utf8]{inputenc}

%opening
\title{WEEK 1: Infinite square well potential, finite square well potential, Wood-Saxon potential}
\author{Ni Fang\\ Kai Wang\\Claudia Gonzalez-Boquera}
\date{TALENT 2016}
\begin{document}

\maketitle
\section{Structure}
The three programs have the same structure in order to find the eigenvalues and eigenvectors 
of the considered system. \\
The main file contains the code that finds the eigenvalues between 0 and 100 MeV (E$_\mathrm{plus}$).
The eigenvalues are found with the bissection method, which includes the numerov 
method in order to calculate the eigenfunction for each eigenvalue. For each
eigenfunction found, we calculate its nodes, and we modify the extremes of the 
bissection method according to them. When the extremes are closer than epsi,
we exit the loop, and put the eigenvalue as the right extreme of the bissection
method. 
Then we recalculate the eigenfunction with the shooting method, with 
right hand side equal to 0, in order to find the convergence, and we normalize it. 

The only thing that changes between the different programs is the potential.
\subsection{Functions and Subroutines}
\begin{itemize}
 \item Function Vpot(xy) :  potential used in each case.
 \item Subroutine Simpson : subroutine to integrate the densities, in order to normalize the 
eigenfunctions. Inputs: N number of points in the mesh, dx space between the points in the mesh,f array
to integrare. Output:res  value of the integration. 
\end{itemize}

\section{Files}
\subsection{Infinite}
\begin{itemize}
 \item global.f90: module that includes the global variables. 
 \item infinite.input: input file (see below).
 \item main.f90: main file that includes the code to find the eigenvalues and eigenfunctions. 
 \item Makefile : makefile to compile the code (see below). 
\end{itemize}
\subsection{Finite}
\begin{itemize}
 \item global.f90: module that includes the global variables. 
 \item finite.input: input file (see below).
 \item main.f90: main file that includes the code to find the eigenvalues and eigenfunctions. 
 \item Makefile : makefile to compile the code (see below). 
 \item Simpson.f90: file that contains the subroutine that integrates using the Simpson method.
\end{itemize}
\subsection{WS}
\begin{itemize}
 \item global.f90: module that includes the global variables. 
 \item WS.input: input file (see below).
 \item main.f90: main file that includes the code to find the eigenvalues and eigenfunctions. 
 \item Makefile : makefile to compile the code (see below). 
 \item Simpson.f90: file that contains the subroutine that integrates using the Simpson method.
\end{itemize}



\section{Input file}
\begin{itemize}
 \item epsi= epsilon error to find the solution. 
 \item h2m = value of $\hbar^2/2m$.
 \item R$_\mathrm{max}$ = in coordinate space, up to where the potential acts.
 \item R$_\mathrm{box}$ = size of our considered and studied system.
 \item h= space between two points in the mesh.
 \item E$_\mathrm{minus}$= left value of the energy when finding the eigenvalues using the bissection method.
 \item E$_\mathrm{plus}$ = right value of the energy when finding the eigenvalues using the bissection method.
 \item V0 = depth of the potential. (used for finite square and WS).
 \item a = difusivity of the Wood-Saxon potential. 
\end{itemize}

\section{Output}
\begin{itemize}
 \item  XYZ.dat files: Output that setore the wavefunctions, where XYZ is the number of the nodes of each wavefunction.
 \item  Console: Output of the eigenvalues according to the number of nodes of the wavefunction, and the name of the file where the
 wavefunction is. 

\end{itemize}

\section{Makefile}
To run the program, there is a Makefile, that compiles all files that form the code. 
To compile, type \textbf{make}, and an executable \textbf{main} is created. To run the code, 
type \textbf{./main}. And to delete all exectuables and outputs, type either \bf{make clean} or
\textbf{make clear}. Makefile has to be changed with vi or vim programs.

\end{document}
